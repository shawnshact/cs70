\documentclass{article}
\begin{document}

	\begin{center}
		\Huge HW 0 ANSWERS
	\end{center}

	\begin{flushleft}
		\huge Sundry: 
	\end{flushleft}

	\begin{itemize}
		\normalsize
		\item Worked on homework independently.
		\item Name(s): Shawn Shacterman
		\item Email(s): shawnshact@berkeley.edu
	\end{itemize}

	\begin{enumerate}
		\huge
		\item Administrivia: 

		\normalsize
		\begin{enumerate}
			\item http://www.eecs70.org
			\item Homework: 10\%, Midterm 1: 25\%, Midterm 2: 25\%, Final: 40\%
		\end{enumerate}

		\huge
		\item Course Policies
	
		\normalsize	
		\begin{enumerate}
			\item Yes. Solutions must always be written up individually.
			\item No. Students are encouraged to work on problems collaboratively, as long as the solution is written independently and proper credit is given to any assisting sources.
			\item Yes. While students have the option of turning to books or online resources, the course policies state that at no time should a student be in possession of another student's solution. Even though Erin did not copy the material verbatim (and the solution did not necessarily belong to another student in the class), she still viewed another person's solution (as opposed to an approach/clarification) which violates the course policies.
			\item Yes. Sending or copying solutions is unacceptable, regardless of any attribution of credit. 
			\item Yes. The course policies state that at no point should a student have possession of another student's solution.  
		\end{enumerate}
	
		\huge
		\item Use of Piazza
		
		\normalsize
		\begin{enumerate}
			\item 13
			\item Depending on the difficulting of proving Theorem XYZ, I would probably encourage the student to go to office hours rather than posting on Piazza (5 minute test). Additionally, posts on Piazza are supposed to be narrow and precise, and it is not the job of TAs or fellow students to rewrite the course notes. I might encourage the student to specifically refer to the part of the proof they are struggling on. 
		\end{enumerate}
		
		\huge
		\item  \LaTeX
		
		\normalsize
		\begin{enumerate}
			\item $\forall x \exists y ((P(x) \land Q(x,y)) \Longrightarrow x \leq \sqrt{y} $
			\item $\displaystyle \sum_{i=0}^{k} {i} = \frac{k(k+1)}{2} $
		\end{enumerate}
	\end{enumerate}
	
\end{document}

